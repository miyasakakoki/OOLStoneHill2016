\documentclass{jsarticle}
\usepackage{listings,jlisting}
\usepackage[dvipdfmx]{graphicx}

\begin{document}

\title{沖縄オープンラボラトリ SDN/クラウド プログラムコンテスト2016 応募}
\maketitle

\section{参加メンバー}
\begin{itemize}
	\item 代表者
		\begin{description}
			\item[氏名] 宮坂 虹槻 (みやさか こうき)
			\item[所属] 神戸情報大学院大学
			\item[E-Mail] s15006@st.kic.ac.jp
		\end{description}
	\item メンバー
		\begin{description}
			\item[氏名] 石原 真太郎 (いしはら しんたろう)
			\item[所属] 京都産業大学大学院
			\item[E-Mail] shintaro.stonefield@gmail.com
		\end{description}
\end{itemize}

\section{テーマ}
	SDNを用いたIoTデバイスの相互通信の簡略化




\section{Abstract}
IoTでは,例えば温湿度センサなどからクラウドにデータが蓄積され,データ解析やアクチュエータにより温度調節などの結果が反映される.それらのノード群はインターネットを介して相互に通信しており,様々なサービスが展開される.しかし,インターネットを介する以上,IPアドレスや通信相手の設定などが全ノードに必要となり,ノードが増えるごとに設定の手間が増える.そのため,ノードの通信設定の簡略化に着目し,SDNを利用したWebUIによる解決を目指す.これにより,例えば,ビルの中にセンサばらまく際に,ノードの管理コストを削減しつつ,集中制御できる.


\section{Internet of Things(IoT)}
	近年,モノのインターネットとも呼ばれるIoTが注目されている.
	IoTのアーキテクチャは,図\ref{fig:1}のように物理現象を計測するセンサ,計測結果を解析するロジック,それにより何かをもたらすアクチュエータから成り,それぞれがインターネットを介して,相互通信することにより,様々な自動化が図られている.
	例えば,センサがクラウドにデータを蓄積し,そのデータを利用してアクチュエータが現実に何らかをもたらす.そして,それらのノード群がインターネットを介して相互に通信し,サービスが展開される.
	様々なことへの応用が考えられており,期待されている分野である.
	例えば,以下のように応用されている.
	\begin{itemize}
		\item 京都市営バス?
		\item
		\item
	\end{itemize}
	しかし,インターネットを介して,IoTデバイスを相互通信させるには,エンドノードであるすべてのデバイスにIPアドレスの設定,通信相手の設定が必要になり,具体的には以下のような課題があげられる.
	\begin{itemize}
		\item IoTデバイス本体の通信を始めるための設定
		\item IoTデバイス同士が相互通信するための設定
		\item IoTデバイスごとの設定ファイルの差異
		\item IoTデバイス数の増加による負担
	\end{itemize}
	そこで,それら通信の設定の課題をネットワーク側で吸収し,各IoTデバイスが通信し始めるための初期設定を不要にしようと考えた.
	具体的には図\ref{fig:2}の様なアーキテクチャを考えており,MACアドレスを用いた各デバイスの識別により,ひとつひとつのIoTデバイスへの個別のIPアドレスを割り当てをなくし,Ehernetヘッダ,IPヘッダに含まれ>る宛先アドレスを書き換える事により,個々のIoTデバイスへの通信相手の設定をなくす.
	IoTデバイスは,無線によりSDNスイッチに接続し,デフォルトゲートウェイにデータを送る.
	SDNスイッチは,基本的に全てのパケットを破棄し,ユーザーが図\ref{fig:3}のようにWebUIからデータの送り先を設定すると,SDNコントローラを通して,SDNスイッチにMACアドレス,IPアドレスを書き換えるフローが入る.
	そして,SDNスイッチはフローに従いパケットを書き換えた後,送出する.
	これにより,WebUIから簡単に送り先の設定や変更ができるようになる.

\section{現時点で取り組んでいる内容}
	現時点での進捗は以下のとおりである。
	\begin{itemize}
		\item MACアドレスを使ってIoTデバイスを識別し,IPアドレスの設定を解決\\
			ただし,コントローラーでホストの情報を管理しておらず,宛先IPアドレスを固定にしている.
			そのため,IoTデバイスのIPアドレスは同じでなくてはならない.
		\item IoTデバイス同士が相互通信するための設定の一部を解決\\
			IoTデバイスからUDPで送られたデータのみ宛先を変更することができる.
			複数の送り先を設定することはできない.
			大量のIoTデバイスの設定の簡略化が解決できていない.
	\end{itemize}

\section{マイルストーン}
	\begin{itemize}
		\item IPアドレスの設定の簡略化の為のIoTデバイスの情報の管理
		\item
		つなげて、サービスを作る  誰がどう嬉しいのか
	\end{itemize}
	MACアドレスでの各デバイスの識別とIPヘッダの書き換えによる「デバイス本体の通信を始めるための設定」の解決
		デバイスのIPアドレスを任意のアドレスに設定した場合でも動くようにする ->  IoTデバイスの情報の管理
		デフォルトゲートウェイ宛へのIPヘッダの書き換えによるARP解決 -> ARP解決方法の変更
		MACアドレスの自動取得,コントローラでのMACアドレスの管理 -> IoTデバイス情報の管理
	Ethernetヘッダ,IPヘッダの書き換えによる「2.デバイス同士が相互通信するための設定」の解決
		UDPを使用した複数の送り先への対応(コントローラとの連携方法)-> 複数の送り先への対応
		フローの削除・変更の実装整理(コントローラとの連携方穂う) ->
	IPヘッダの書き換えによる、「3.IoTデバイスごとの設定ファイルの差異」の解決
		設定ファイルが違うという事の説明(複数種類のセンサを使用したデモ?)
	WebUIによる「4.デバイス数の増加による負担」の軽減
	チェックボックスによる簡単な選択
		(デバイスのグループ化)-> WebUIの改良
		-> 大量のデバイスへの一括設定

\end{document}
