\documentclass{jsarticle}
\usepackage{listings,jlisting}
\usepackage[dvipdfmx]{graphicx}

\begin{document}

\title{沖縄オープンラボラトリ SDN/クラウド プログラムコンテスト2016 応募}
\maketitle

\section{参加メンバー}
\begin{itemize}
	\item 代表者
		\begin{description}
			\item[氏名] 宮坂 虹槻 (みやさか こうき)
			\item[所属] 神戸情報大学院大学
			\item[E-Mail] s15006@st.kic.ac.jp
		\end{description}
	\item メンバー
		\begin{description}
			\item[氏名] 石原 真太郎 (いしはら しんたろう)
			\item[所属] 京都産業大学大学院
			\item[E-Mail] shintaro.stonefield@gmail.com
		\end{description}
\end{itemize}

\section{テーマ}
	SDNを用いたIoTデバイスの相互通信の簡略化

\section{背景}
	Internet of Thingsの略で,物のインターネットとも呼ばれる.\\
	IoTのアーキテクチャは,図\ref{fig:1}のように物理現象を計測するセンサ,計測結果を解析するロジック,それにより何かをもたらすアクチュエータから成り,それぞれがインターネットに接続し,相互通信することにより,様々な自動化が図られている. 様々なことへの応用が考えられており,期待されている分野である. 例えば,以下のように応用されている.
	\begin{itemize}
		\item 
		\item 
		\item 
	\end{itemize}

\section{IoTの課題}

\section{関連}

\section{SDNを用いた解決}

\section{特徴}

\section{アーキテクチャ}

\end{document}

